% !TEX encoding = UTF-8 Unicode

% ------------------------------------------------------------------------------------------------------
%	Formatvorlage für wissenschaftliche Arbeiten (Diplomarbeit, Bachelorarbeit, Masterarbeit, Seminararbeit, Projektarbeit, Dokumentation)
% ------------------------------------------------------------------------------------------------------
%	ursprünglich erstellt von Stefan Macke, 24.04.2009
%	http://blog.stefan-macke.de
%
%	erweitert von Felix Rupp, 21.05.2013
%	http://www.felixrupp.com/
%
%
%
% erweitert und angepasst von Moritz Rupp
% moritz.rupp@gmail.com
% http://www.moritzrupp.de
%
% Version 2.0.2
% Datum: 27.05.2014
%
% An THM Standarts angepasst von Florian Siepe
% flo.siepe@gmail.com
% Datum: 19.4.2017


% Dokumentenkopf ---------------------------------------------------------------------------------------
%   Diese Vorlage basiert auf "scrreprt" aus dem koma-script.
% ------------------------------------------------------------------------------------------------------
\documentclass[
    12pt, % Schriftgröße
    DIV10, % Änderung der Größe des Satzspiegels (bedruckbarer Bereich einer Seite), nur in Verbindung mit koma-script verwendbar
    ngerman, % für Umlaute, Silbentrennung etc.
    a4paper, % Papierformat
    oneside, % einseitiges Dokument
    titlepage, % es wird eine Titelseite verwendet
    parskip=half, % Abstand zwischen Absätzen (halbe Zeile)
    headings=normal, % Größe der Überschriften verkleinern
    listof=totoc, % Verzeichnisse im Inhaltsverzeichnis aufführen
    bibliography=totoc, % Literaturverzeichnis im Inhaltsverzeichnis aufführen
    index=totoc, % Index im Inhaltsverzeichnis aufführen
    captions=tableheading, % Beschriftung von Tabellen unterhalb ausgeben
    final % Status des Dokuments (final/draft)
]{scrreprt}

% UTF8 und T1 Fontencoding -----------------------------------------------------------------------------
\usepackage[utf8]{inputenc}
\usepackage[T1]{fontenc}


% Meta-Informationen -----------------------------------------------------------------------------------
%   Informationen über das Dokument, wie z.B. Titel, Autor, Matrikelnr. etc
%   werden in der Datei Meta.tex definiert und können danach global
%   verwendet werden.
% ------------------------------------------------------------------------------------------------------
% !TEX encoding = UTF-8 Unicode
% !TEX root =  Arbeit.tex

% Meta-Informationen ------------------------------------------------------------------------------------
%   Definition von globalen Parametern, die im gesamten Dokument verwendet
%   werden können (z.B auf dem Deckblatt etc.).
%
%   ACHTUNG: Wenn die Texte Umlaute oder ein Esszet enthalten, muss der folgende
%            Befehl bereits an dieser Stelle aktiviert werden:
%            \usepackage[latin1]{inputenc}
% -------------------------------------------------------------------------------------------------------
\newcommand{\titel}{Mein Titel}
\newcommand{\titelDeckblatt}{Mein Titel für das Deckbaltt}

\newcommand{\untertitel}{Untertitel der Arbeit}
\newcommand{\untertitelDeckblatt}{Untertitel der Arbeit\\ f\"ur das Deckblatt}

\newcommand{\art}{Typ des Dokumentes}
\newcommand{\fachgebiet}{Mein Fachgebiet}
\newcommand{\studienbereich}{Studiengang}

\newcommand{\autor}{Mein Name}
\newcommand{\autorStrasse}{Meine Straße}
\newcommand{\autorOrt}{Mein Ort}

\newcommand{\keywords}{}

\newcommand{\matrikelnr}{Matrikelnummer}
%\newcommand{\kurs}{KursABC1}

\newcommand{\hochschulbetreuer}{Hochschulbetreuer}
\newcommand{\unternehmensbetreuer}{Unternehmensbetreuer}

\newcommand{\jahr}{Jahr}
\newcommand{\abgabeDatum}{Abgabedatum}

\newcommand{\hochschule}[1][\empty]{Technische Hochschule Mittelhessen}
\newcommand{\firma}{Meine Firma}
\newcommand{\plz}{PLZ der Firma}
\newcommand{\ort}{Ort der Firma}
\newcommand{\strasse}{Straße der Firma}

\newcommand{\logoTHM}{Bilder/logos/Logo_THM.jpg}
\newcommand{\logoSTP}{Bilder/logos/Logo_StudiumPlus.jpg}
\newcommand{\logoFirma}{Bilder/Logo.pdf}

\newcommand{\creator}{texmaker 4.1.1}

\newcommand{\bibliographyheading}{Literaturverzeichnis}

\newcommand{\TableCellPadding}{1.2}


% benötigte Packages -----------------------------------------------------------------------------------
%   LaTeX-Packages, die benötigt werden, sind in die Datei Packages.tex
%   "ausgelagert", um diese Vorlage möglichst übersichtlich zu halten.
% ------------------------------------------------------------------------------------------------------
% !TEX encoding = UTF-8 Unicode
% !TEX root =  Arbeit.tex

% Anpassung des Seitenlayouts ---------------------------------------------------
%   siehe Seitenstil.tex
% -----------------------------------------------------------------------------------------
\usepackage[
    automark, % Kapitelangaben in Kopfzeile automatisch erstellen
    headsepline, % Trennlinie unter Kopfzeile
    ilines, % Trennlinie linksbündig ausrichten
    plainfootsepline
]{scrpage2}

% Verhindert bestimmte Warnungen
\usepackage{scrhack}

% Alle Figuren durchnummerieren
\usepackage{chngcntr}
\counterwithout{figure}{chapter}
\counterwithout{table}{chapter}

\usepackage[titles]{tocloft}                  %titles für seperate Seiten
\renewcommand{\thefigure}{\bfseries\arabic{figure}}
\renewcommand{\thetable}{\bfseries\arabic{table}}
\renewcommand{\cfttabpresnum}{\textbf{Tab.}  } 
\renewcommand{\cftfigpresnum}{\textbf{Abb.}  }
\renewcommand{\cftfigaftersnum}{\textbf{:}}
\renewcommand{\cfttabaftersnum}{\textbf{:}}
\setlength{\cftfignumwidth}{2cm}                     
\setlength{\cfttabnumwidth}{2,5cm}                                           
\setlength{\cftfigindent}{0cm}                                                     
\setlength{\cfttabindent}{0cm} 

% Anpassung an Landessprache -------------------------------------------------
\usepackage[ngerman]{babel}

% Verhindern der Fußzeile für Verzeichnisse
% Umlaute ------------------------------------------------------------------------------
%   Umlaute/Sonderzeichen wie äüöß direkt im Quelltext verwenden (CodePage).
%   Erlaubt automatische Trennung von Worten mit Umlauten.
%   Für Umlaute siehe Hauptdokument Zeile 41
% -----------------------------------------------------------------------------------------
\usepackage{textcomp} % Euro-Zeichen etc.


% Schrift --------------------------------------------------------------------------------
\usepackage{lmodern} % bessere Fonts
\usepackage{relsize} % Schriftgröße relativ festlegen


% Bessere Unterstreichungen ---------------------------------------------
\usepackage[normalem]{ulem}


% Grafiken -----------------------------------------------------------------------------
% Einbinden von JPG-Grafiken ermöglichen
\usepackage[dvips,final]{graphicx}
% hier liegen die Bilder des Dokuments
\graphicspath{{Bilder/}}
\usepackage{picture}
\setkomafont{captionlabel}{\normalsize\bfseries} 


% Befehle aus AMSTeX für mathematische Symbole z.B. \boldsymbol \mathbb
\usepackage{amsmath,amsfonts}


% Eurozeichen benutzen ------------------------------------------------------------
\usepackage{eurosym}


% für Index-Ausgabe mit \printindex -----------------------------------------------
\usepackage{makeidx}


% Einfache Definition der Zeilenabstände und Seitenränder etc. ------------
\usepackage{setspace}
\usepackage{geometry}

% zum Umfließen von Bildern ---------------------------------------------------------
\usepackage[vflt]{floatflt}
\usepackage{subfig}
\usepackage{tocbasic}

% zum Einbinden von Programmcode -----------------------------------------------
\usepackage{listings}
\usepackage{xcolor} 
\definecolor{hellgelb}{rgb}{1,1,0.9}
\definecolor{colKeys}{rgb}{0.8,0,0.5}
\definecolor{colIdentifier}{rgb}{0.6,0,0.3}
\definecolor{colComments}{rgb}{0,0.5,0}
\definecolor{colString}{rgb}{0,0,1}

\lstset{
    float=htbp,
		captionpos=b,
    basicstyle=\ttfamily\color{black}\small\smaller,
    identifierstyle=,%\color{colIdentifier},
    keywordstyle=\color{colKeys}\bfseries,
    stringstyle=\color{colString},
    commentstyle=\color{colComments},
    columns=flexible,
    tabsize=4,
    frame=single,
    extendedchars=true,
    showspaces=false,
    showstringspaces=false,
    numbers=left,
    numberstyle=\tiny,
    breaklines=true,
    backgroundcolor=\color{hellgelb},
    breakautoindent=true,
    escapeinside={(*}{*)},
    literate={Ö}{{\"O}}1 {Ä}{{\"A}}1 {Ü}{{\"U}}1 {ß}{{\ss}}2 {ü}{{\"u}}1 {ä}{{\"a}}1 {ö}{{\"o}}1 {µ}{\textmu}1
 }

% --------------------------------------------------------------------------------------------
% 
% Eigene Definitionen für Quelltext-Stile
%
% Beispieldefinition von CSS:
\lstdefinelanguage{CSS}{
    morestring=[b]',
    morestring=[b]",
    comment=[l]{/*}{*/},
    sensitive=false,
    morekeywords={accelerator,adjust,after,align,attachment,azimuth,background,before,behavior,binding,border,bottom,bottomright,bottomleft,break,caption-side,char,clear,clip,color,colors,cue,cursor,collapse,decoration,direction,display,elevation,empty-cells,family,filter,float,flow,focus,font,grid,height,image,increment,indent,input,inside,ime-mode,include-source,justify,last,layer,left,letter,line,list,margin,marker,marks,max,min,mode,modify,moz,offsett,opacity,orphans,outline,overflow,overflow-X,overflow-Y,overhang,position,position-x,position-y,padding,page,pause,pitch,play-during,position,quotes,radius,range,repeat,replace,reset,richness,right,ruby,set-link-source,select,shadow,size,spacing,speak,speak-header,speak-numeral,speak-punctuation,speech-rate,stress,stretch,style,table-layout,text,transform,text-autospace,text-kashida-space,top,topleft,topright,type,underline,unicode-bidi,use-link-source,user,variant,vertical,visibility,voice-family,volume,white-space,weight,widows,width,word,wrap,word-wrap,writing-mode,z-index,zoom}
}


% URL verlinken, lange URLs umbrechen etc. -------------------------------------
\usepackage{url}


% biblatex einbinden -----------------------------------------------------------------------
\usepackage[
	citestyle=authoryear-icomp,	
	autocite=footnote,
	%citestyle=my-authoryear-ibid,
	bibstyle=authoryear,
	backend=biber,
	autolang=hyphen,
	urldate=comp,
	dateabbrev=false,
	sorting=nyt,
	autopunct=true,
	maxcitenames=1
]{biblatex}
\usepackage[babel,german=quotes]{csquotes}
\DeclareNameAlias{author}{last-first}

     
% Semikolon zwischen den Autoren
\renewcommand*{\multinamedelim}{\addsemicolon\space}

% PDF-Optionen -------------------------------------------------------------------------
\usepackage[
    bookmarks,
    bookmarksopen=true,
    colorlinks=true,
% diese Farbdefinitionen zeichnen Links im PDF farblich aus
    linkcolor=red, % einfache interne Verknüpfungen
    anchorcolor=black,% Ankertext
    citecolor=blue, % Verweise auf Literaturverzeichniseinträge im Text
    filecolor=magenta, % Verknüpfungen, die lokale Dateien öffnen
    menucolor=red, % Acrobat-Menüpunkte
    urlcolor=cyan, 
%
% diese Farbdefinitionen sollten für den Druck verwendet werden (alles schwarz):
%
    linkcolor=black, % einfache interne Verknüpfungen
    anchorcolor=black, % Ankertext
    citecolor=black, % Verweise auf Literaturverzeichniseinträge im Text
    filecolor=black, % Verknüpfungen, die lokale Dateien öffnen
    menucolor=black, % Acrobat-Menüpunkte
    urlcolor=black, 
%
    plainpages=false, % zur korrekten Erstellung der Bookmarks
    pdfpagelabels, % zur korrekten Erstellung der Bookmarks
    hypertexnames=false, % zur korrekten Erstellung der Bookmarks
    linktocpage % Seitenzahlen anstatt Text im Inhaltsverzeichnis verlinken
]{hyperref}
% Befehle, die Umlaute ausgeben, führen zu Fehlern, wenn sie hyperref als Optionen übergeben werden
\hypersetup{
    pdftitle={\titel \untertitel},
    pdfauthor={\autor},
    pdfcreator={\creator},
    pdfsubject={\titel \untertitel},
    pdfkeywords={\keywords},
}

% Glossar und Abkürzungsverzeichnis -------------------------
\usepackage[
	%xindy,
	nonumberlist,	% keine Seitenzahlen anzeigen
	acronym,		% ein Abkürzungsverzeichnis erstellen
	toc,			% Einträge im Inhaltsverzeichnis
	shortcuts		% acronym shortcuts
]{glossaries}

% Paket zum sauberen Einbauen von externen PDF-Dateien -----------------
\usepackage[final]{pdfpages}


% fortlaufendes Durchnummerieren der Fußnoten -------------------------------
\usepackage{chngcntr}


% schönere Tabellen --------------------------------------------------------------------
\usepackage{tabularx}
\usepackage{booktabs}


% für lange Tabellen ---------------------------------------------------------------------
\usepackage{longtable}
\usepackage{ltxtable}
\usepackage{filecontents}
\usepackage{array}
\usepackage{ragged2e}
\usepackage{lscape}


% Rotation von Elementen -------------------------------------------------------
\usepackage{rotating}


% Formatierung von Listen ändern --------------------------------------------------
\usepackage{paralist}
\usepackage{enumitem}


% bei der Definition eigener Befehle benötigt -------------------------------------
\usepackage{ifthen}
\usepackage{forloop}


% definiert u.a. die Befehle \todo und \listoftodos --------------------------------
\usepackage[ngerman]{todonotes}

% Durchstreichen von Textstellen
\usepackage{ulem}

% sorgt dafür, dass Leerzeichen hinter parameterlosen Makros nicht als Makroendezeichen interpretiert werden
\usepackage{xspace}

% Für fette Bibliographieeinträge
\usepackage{xpatch}

% Für mehrspaltige und mehrzeilige Tabellen
\usepackage{multicol}
\usepackage{multirow}

% Bessere Silbentrennung etc.
\usepackage{microtype}

% Blindtext im Inhaltsbereich
\usepackage{blindtext}

% Appendix
\usepackage{appendix}

%% Kapitelanfang nicht auf neuer Seite ----------------------------------------
%\usepackage{etoolbox}
%\makeatletter
%\patchcmd{\chapter}{\if@openright\cleardoublepage\else\clearpage\fi}{}{}{}
%\makeatother



% Erstellung eines Index und Abkürzungsverzeichnisses/Glossars aktivieren ------------------------------
\makeindex{}


% Kopf- und Fußzeilen, Seitenränder etc. ---------------------------------------------------------------
\input{Seitenstil}


% eigene Definitionen für Silbentrennung ---------------------------------------------------------------
% !TEX encoding = UTF-8 Unicode
% !TEX root =  Arbeit.tex

% Trennvorschläge im Text werden mit \" angegeben
% untrennbare Wörter und Ausnahmen von der normalen Trennung können in dieser
% Datei mittels \hyphenation definiert werden

% \hyphenation{Mit-ar-bei-ter}


% eigene LaTeX-Befehle ---------------------------------------------------------------------------------
% !TEX encoding = UTF-8 Unicode
% !TEX root =  Arbeit.tex
% Eigene Befehle und typographische Auszeichnungen für diese


% einfaches Wechseln der Schrift, z.B.: \changefont{cmss}{sbc}{n} ---------------------------------------
\newcommand{\changefont}[3]{\fontfamily{#1} \fontseries{#2} \fontshape{#3} \selectfont}


% Abkürzungen mit korrektem Leerraum --------------------------------------------------------------------
\newcommand{\ua}{\mbox{u.\,a.\ }}
\newcommand{\zB}{\mbox{z.\,B.\ }}
\newcommand{\dahe}{\mbox{d.\,h.\ }}
\newcommand{\Vgl}{Vgl.\ }
\newcommand{\bzw}{bzw.\ }
\newcommand{\evtl}{evtl.\ }
\newcommand{\zb}{z.B.\ }

\newcommand{\Abbildung}[1]{Abbildung~\ref{fig:#1}}

\newcommand{\bs}{$\backslash$}


% erzeugt ein Listenelement mit fetter Überschrift ------------------------------------------------------
\newcommand{\itemd}[2]{\item{\textbf{#1}}\\{#2}}


% zum Ausgeben von Autoren
\newcommand{\AutorName}[1]{\textsc{#1}}
\newcommand{\Autor}[1]{\AutorName{\citeauthor{#1}}}


% verschiedene Befehle um Wörter semantisch auszuzeichnen -----------------------------------------------
\newcommand{\NeuerBegriff}[1]{\textit{#1}}


% Beträge mit Währung -----------------------------------------------------------------------------------
\newcommand{\Betrag}[2][general]{#2\,\ifthenelse{\equal{#1}{dollar}}{\$}{}\ifthenelse{\equal{#1}{euro}}{€}{}\ifthenelse{\equal{#1}{yen}}{¥}{}\ifthenelse{\equal{#1}{cent}}{¢}{}\ifthenelse{\equal{#1}{pound}}{£}{}\ifthenelse{\equal{#1}{peso}}{₱}{}\ifthenelse{\equal{#1}{baht}}{฿}{}\ifthenelse{\equal{#1}{franc}}{₣}{}\ifthenelse{\equal{#1}{lira}}{₤}{}\ifthenelse{\equal{#1}{drachma}}{₯}{}\ifthenelse{\equal{#1}{pfennig}}{₰}{}\ifthenelse{\equal{#1}{general}}{¤}{}}


% Sonstiges ---------------------------------------------------------------------------------------------
\newcommand{\Eingabe}[1]{\texttt{#1}}
\newcommand{\Code}[1]{\texttt{#1}}
\newcommand{\Datei}[1]{\texttt{#1}}


% Beschriftung von Tabellen und Bildern ändern ----------------------------------------------------------
\addto\captionsngerman{
	\renewcommand{\figurename}{Abb.}
	\renewcommand{\tablename}{Tab.}
}


% Spaltendefinition rechtsbündig mit definierter Breite -------------------------------------------------
\newcolumntype{w}[1]{>{\raggedleft\hspace{0pt}}p{#1}}


% Linksbündige Tabellenspalten mit tabularx -------------------------------------------------------------
\newcolumntype{y}[1]{>{\RaggedRight\arraybackslash\hsize=#1\hsize}X}



% Literaturverzeichnis ---------------------------------------------------------------------------------
%   Das Literaturverzeichnis wird aus der BibTeX-Datenbank "Bibliographie.bib"
%   erstellt.
% ------------------------------------------------------------------------------------------------------
\bibliography{Literatur} % Aufruf: biber Bibliographie

% Um verschiede Unterkapitel im Literaturverzeichnis zu erstellen, wird der Befehl defbibheading verwendet
\defbibheading{internet}{\section*{Internetquellen}}
\defbibheading{literatur}{\section*{Fachliteratur}}


% Glossar --------------------------------------------------------------------------------
\makeglossaries
\loadglsentries{Inhalt/Glossar.tex}

% Hinzufügen der nicht benutzten Glossareinträge (falls noch eine Abkürzung vorhanden ist und nur diese verwendet wird)
%\glsadd{hapi}


% Das eigentliche Dokument -----------------------------------------------------------------------------
%   Der eigentliche Inhalt des Dokuments beginnt hier. Die einzelnen Seiten
%   und Kapitel werden in eigene Dateien ausgelagert und hier nur inkludiert.
% ------------------------------------------------------------------------------------------------------
\begin{document}


% auch subsubsections nummerieren ----------------------------------------------------------------------
\setcounter{secnumdepth}{3}
% Nummerierungsebenen im Inhaltsverzeichnis
\setcounter{tocdepth}{2}


% Deckblatt und Abstract ohne Seitenzahl ---------------------------------------------------------------
\ofoot{}
% !TEX encoding = UTF-8 Unicode
% !TEX root =  Arbeit.tex

\thispagestyle{plain}
\begin{titlepage}


%Dieses Dokument muss mittels import-Befehl in das Hauptdokument eingebunden werden
%Benötigtes Paket für Grafik: \usepackage{graphicx}

\begin{addmargin}[-0.5cm]{0cm}
\thispagestyle{empty} %% ohne Kopf- und Fusszeile, Seitennummer etc.

\begin{figure}[h]
  \centering
  		\includegraphics[height=1.5cm,keepaspectratio]{\logoTHM}
  		\hfill
  		\includegraphics[height=1.79cm,keepaspectratio]{\logoSTP}
\end{figure}
		
	\begin{center}	
	
			\vspace{15mm}		
			\large
			\textbf{Praxisphasenbericht}
			\large
			
			Praxisphasenbericht in der 1. Praxisphase
%			\textbf{Projektstudiumsbericht} \\
%			Projektstudium
	\end{center}

			\vspace{1.0cm}
			

			Thema:

	\begin{center}
    
			\large
			%\vspace{0.1cm}
			
			\titelDeckblatt
			
			\large
			
			\normalsize


			\vspace{10mm}
									
			\small
			
				\end{center}
			
			\begin{tabular}[h]{ll}
  				Vorgelegt von: 	&	\autor	\\
  								&	\autorStrasse	\\
  								&	\autorOrt	\\
%  								&	Herrn Andreas Schwertmann\\\\
  				Matrikelnummer: &  	\matrikelnr	\\

  				Eingereicht bei:&	\\
  				Hochschulbetreuer:	&	\hochschulbetreuer	\\
  				Fachbetreuer:	&	\unternehmensbetreuer	\\
  				Unternehmen:	& 	\firma \\
  								& 	\strasse \\
  								& 	\plz \ \ort \\
  								\\
  				Eingereicht am:	& 	\abgabeDatum%\today
  				%Eingereicht am:	& \the\day.\the\month.\the\year
			\end{tabular}
					
					
			\vspace{10mm}
			
			
			\begin{flushright}
					\includegraphics[height=1.5cm,keepaspectratio]{\logoFirma}
			\end{flushright}
							

	 
% 		\vfill


\end{addmargin}


\end{titlepage}


% Sperrvermerk ----------------------------------------------------------------------------
% !TEX encoding = UTF-8 Unicode
% !TEX root =  Arbeit.tex

\chapter*{Sperrvermerk}

\thispagestyle{empty}

Der vorliegende {\art} beinhaltet interne vertrauliche Informationen der  \firma{}. Die Weitergabe des Inhaltes der Arbeit und eventuell beiliegender Zeichnungen und Daten im Gesamten oder in Teilen ist grunds\"atzlich untersagt. Es dürfen keinerlei Kopien oder Abschriften - auch in digitaler Form - gefertigt werden. Ausnahmen bed\"urfen der schriftlichen Genehmigung der  \firma{}.


%  Abstract ------------------------------------------------------------------------------------------
%% !TEX encoding = UTF-8 Unicode
% !TEX root =  ../Arbeit.tex

\chapter*{Abstract}
\label{cha:Abstract}

\thispagestyle{empty}


\blindtext[2]

\vspace{\fill}

\begin{description}
	\item[Stichworte:] \keywords
\end{description}

% Liste mit Todos -------------------------------------------------------------------------------------
%\phantomsection
%\markboth{Liste der noch zu erledigenden Punkte}{Liste der noch zu erledigenden Punkte}
%\listoftodos
%\newpage


% Seitennummerierung -----------------------------------------------------------------------------------
%   Vor dem Hauptteil werden die Seiten in großen römischen Ziffern 
%   nummeriert.
% ------------------------------------------------------------------------------------------------------
\ofoot{\pagemark \\[4ex]}
\pagenumbering{Roman}

% Inhaltsverzeichnis --------------------------------------------------------------------------------------
\phantomsection{} % Sorgt für korrekte Aufnahme des Inhaltsverzeichnisses in das Inhaltsverzeichnis
\addcontentsline{toc}{chapter}{Inhaltsverzeichnis}
\tableofcontents{}


%%
% Anzeigen des Abkürzungsverzeichnisses und des Glossars -------------------------------------
%%
% Glossar
%\newpage
\setglossarypreamble[acronym]{Allgemeine Abkürzungen, wie „z.B.“, „etc.“, „usw.“ oder gängige Abkürzungen in Fußnoten, wie z.B. „vgl.“, usw. wurden explizit nicht im Abkürzungsverzeichnis aufgeführt.\vspace{20mm}}

\renewcommand{\lstlistlistingname}{Abkürzungsverzeichnis}
\renewcommand{\glsnamefont}[1]{\makefirstuc{#1}} % Druckt dern ersten Buchstaben jeden Eintrages groß
\printglossary[style=altlist,type=main,title={Glossar}]
\label{cha:Glossar}

% Abkürzungen
%\newpage
\markboth{Abkürzungsverzeichnis}{Abkürzungsverzeichnis}
\printglossary[style=listdotted, type=\acronymtype, title={Abkürzungsverzeichnis}]


\label{cha:Abkuerzungsverzeichnis}


% Restliche Verzeichnisse ------------------------------------------------------------------------------
\listoffigures{} % Abbildungsverzeichnis
\listoftables{} % Tabellenverzeichnis

% arabische Seitenzahlen im Hauptteil ------------------------------------------------------------------
\clearpage{}
\pagenumbering{arabic}


% die Inhaltskapitel werden in "Inhalt.tex" inkludiert -------------------------------------------------
% !TEX encoding = UTF-8 Unicode
% !TEX root =  Arbeit.tex

% Hier können die einzelnen Kapitel inkludiert werden. Sie müssen in den 
% entsprechenden .TEX-Dateien vorliegen. Die Dateinamen können natürlich 
% angepasst werden.

%% !TEX encoding = UTF-8 Unicode
% !TEX root =  ../Arbeit.tex

\chapter{Einleitung}
\label{cha:Einleitung}

%%%%%%%%%%%%%%%%%%%%%%%%%%%%%%%%%%%%%%%%%%%%%%%%%%%%%%%%%%%%%%

%\section{Beispiel des Glossars}
%\label{sec:BspGlossar}
%
%\gls{Glossareintrag} ist ein beispielhafter Glossareintrag, während \gls{AGPL} eine Abkürzung darstellt.
%
%\blindtext[5] \autocite{Testautor2002}
%
%\section{Blindtext}
%\label{sec:Blindtext}
%
%\blindtext[2] \autocite{Author2013}
%
%\section{Listen}
%\label{sec:Listen}
%
%\blindtext
%
%\blinditemize[5]
%
%\blindtext[2]
%% !TEX encoding = UTF-8 Unicode
% !TEX root =  ../Arbeit.tex

\chapter{Hauptteil}
\label{cha:Hauptteil}

%% !TEX encoding = UTF-8 Unicode
% !TEX root =  ../Arbeit.tex

\chapter{Fazit und Ausblick}
\label{cha:fazit}




\clearpage{}
\pagenumbering{Roman}
\setcounter{page}{6} %%% Dieser Pagecounter muss entsprechend der verbrauchten Seiten im Inhaltsverzeichnis angepasst werden. Endet das IHV bei Seite III, so muss hier 4 eingetragen werden


% Literaturverzeichnis anzeigen
\phantomsection
\markboth{Literaturverzeichnis}{Literaturverzeichnis}
\addcontentsline{toc}{chapter}{Literaturverzeichnis}
\chapter*{Literaturverzeichnis}

\printbibliography[notkeyword=Online, heading=literatur] % Alle keywords aufzählen, die nicht gedruckt werden sollen
\newpage
\begingroup
	\raggedright
	\sloppy
	\newrefcontext[sorting=nty]
	\printbibliography[keyword=Online, heading=internet, sorting=nty] % Keywords aufzählen, die hier gedruckt werden sollen
\endgroup

% Selbständigkeitserklärung ----------------------------------------------------------------------------
\addcontentsline{toc}{chapter}{Versicherung}

% !TEX encoding = UTF-8 Unicode
% !TEX root =  Arbeit.tex

\chapter*{Versicherung}
\thispagestyle{empty}

Ich, \autor, Matrikel-Nr.\ \matrikelnr, versichere hiermit, dass ich meinen \art\xspace mit dem Thema

\begin{quote}
\textit{\titel} %\textit{\untertitel}
\end{quote}

selbständig verfasst und keine anderen als die angegebenen Quellen und Hilfsmittel benutzt habe, wobei ich alle wörtlichen und sinngemäßen Zitate als solche gekennzeichnet habe. Die Arbeit wurde bisher keiner anderen Prüfungsbehörde vorgelegt und auch nicht veröffentlicht.

\vspace{8ex}

\ort, den \abgabeDatum

\vspace{6ex}


\rule[-0.2cm]{5cm}{0.5pt}

\textsc{\autor} 

% Index ------------------------------------------------------------------------------------------------
%   Zum Erstellen eines Index, die folgende Zeile auskommentieren.
% ------------------------------------------------------------------------------------------------------
%\printindex


% Anhang -----------------------------------------------------------------------------------------------
%   Die Inhalte des Anhangs werden analog zu den Kapiteln inkludiert.
%   Dies geschieht in der Datei "Anhang.tex".
% ------------------------------------------------------------------------------------------------------
%\appendix
%
%\makeatletter
%\renewcommand{\@chap@pppage}{%
%	\clear@ppage
%	\thispagestyle{scrheadings}%
%	\if@twocolumn\onecolumn\@tempswatrue\else\@tempswafalse\fi
%	\null\vfil
%	\markboth{Appendices}{Appendices}%
%	{\centering
%		\interlinepenalty \@M
%		\normalfont
%		\Huge \bfseries \appendixpagename\par}%
%	\if@dotoc@pp
%		\addappheadtotoc
%	\fi
%}
%\makeatother
%
%\clearpage{}
%\addappheadtotoc
%\appendixpage
%
%\pagenumbering{roman}
%
%% !TEX encoding = UTF-8 Unicode
% !TEX root =  Arbeit.tex

\chapter{Beiliegende CD}
\label{cha:BeiliegendeCd}

\begin{enumerate}
	\item Die gesamte Bachelorarbeit als PDF-Datei
	\item Alle verwendeten Online-Quellen als PDF-Ausdruck
	\item Sonstige Quelltexte
	\item \ldots
\end{enumerate}

\end{document}
