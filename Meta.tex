% !TEX encoding = UTF-8 Unicode
% !TEX root =  Arbeit.tex

% Meta-Informationen ------------------------------------------------------------------------------------
%   Definition von globalen Parametern, die im gesamten Dokument verwendet
%   werden können (z.B auf dem Deckblatt etc.).
%
%   ACHTUNG: Wenn die Texte Umlaute oder ein Esszet enthalten, muss der folgende
%            Befehl bereits an dieser Stelle aktiviert werden:
%            \usepackage[latin1]{inputenc}
% -------------------------------------------------------------------------------------------------------
\newcommand{\titel}{Mein Titel}
\newcommand{\titelDeckblatt}{Mein Titel für das Deckbaltt}

\newcommand{\untertitel}{Untertitel der Arbeit}
\newcommand{\untertitelDeckblatt}{Untertitel der Arbeit\\ f\"ur das Deckblatt}

\newcommand{\art}{Typ des Dokumentes}
\newcommand{\fachgebiet}{Mein Fachgebiet}
\newcommand{\studienbereich}{Studiengang}

\newcommand{\autor}{Mein Name}
\newcommand{\autorStrasse}{Meine Straße}
\newcommand{\autorOrt}{Mein Ort}

\newcommand{\keywords}{}

\newcommand{\matrikelnr}{Matrikelnummer}
%\newcommand{\kurs}{KursABC1}

\newcommand{\hochschulbetreuer}{Hochschulbetreuer}
\newcommand{\unternehmensbetreuer}{Unternehmensbetreuer}

\newcommand{\jahr}{Jahr}
\newcommand{\abgabeDatum}{Abgabedatum}

\newcommand{\hochschule}[1][\empty]{Technische Hochschule Mittelhessen}
\newcommand{\firma}{Meine Firma}
\newcommand{\plz}{PLZ der Firma}
\newcommand{\ort}{Ort der Firma}
\newcommand{\strasse}{Straße der Firma}

\newcommand{\logoTHM}{Bilder/logos/Logo_THM.jpg}
\newcommand{\logoSTP}{Bilder/logos/Logo_StudiumPlus.jpg}
\newcommand{\logoFirma}{Bilder/Logo.pdf}

\newcommand{\creator}{texmaker 4.1.1}

\newcommand{\bibliographyheading}{Literaturverzeichnis}

\newcommand{\TableCellPadding}{1.2}