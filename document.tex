% !TeX root = document.tex
\documentclass[12pt,a4paper]{THM}
\usepackage[utf8]{inputenc}
\usepackage{amsmath}
\usepackage{ngerman}
\usepackage{amsfonts}
\usepackage{amssymb}
\usepackage{graphicx}
\usepackage{blindtext}
\usepackage{biblatex}

\documentType{Praxisphasenbericht}{SS 2018}

\author{Dein Name}{Deine PLZ}{Dein Ort}{Deine Straße}{Deine Matnr}

\title{Dein Titel}

\companysupervisor{Der Unternehmensbetreuer}
\unisupervisor{Der Prof}
\company{Deine Firma}{PLZ der Firma}{Ort der Firma}{Straße der Firma}{resources/logo.png}
\signature{signature}{-8ex}{2ex}

\begin{document}

\maketitle
\lockMark
\pagenumbering{Roman}
\addcontentsline{toc}{chapter}{\contentsname}
\tableofcontents
\setcounter{page}{1}
\newpage
\addcontentsline{toc}{chapter}{\listfigurename}
\listoffigures
\newpage
\addcontentsline{toc}{chapter}{\listtablename}
\listoftables
\newpage
\addcontentsline{toc}{chapter}{Abkürzungsverzeichnis}
\chapter*{Abkürzungsverzeichnis}
\commonAcronyms
\newpage
\pagenumbering{arabic}
\setcounter{page}{1}

\chapter{Ein Kapitel}
\section{Ein Abschnitt}
\blindtext
\blindtext\info[inline]{This is very important}
\blindtext
\blindtext

\blindtext\unsure{I am not sure}

\blindtext

\blindtext

\blindtext

\printbibliography[title=Literaturverzeichnis]


\pagenumbering{Roman}
\setcounter{page}{5}
\addcontentsline{toc}{chapter}{Literaturverzeichnis}
\makeinsurance
\end{document}
